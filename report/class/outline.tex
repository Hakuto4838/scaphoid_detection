\graphicspath{{./fig/}}
\section{Outline}
以下說明提交的程式碼檔案概要

    \paragraph{ui.py}
    \begin{itemize}
        \item for Demo
        \begin{figure}[h]
            \centering
            \includegraphics[width=0.9\textwidth]{ui.png}
            \label{figui}
            \caption{圖形介面展示}
        \end{figure}
        \item Load Folder : 選擇 image dir,讀取後 Image Display 區域將顯示第一張圖片,並以 [index/Totol image number] 顯示目前的資料夾資料總數、目前影像名稱等等
        \item Previous, Next : 切換上/下一張影像,切換後會自動執行 Detection,在區域中以紅框標示
        \item Load label : 設定舟骨 label 資料夾,若有設定,且路徑中存在與當前影像相同名稱的json,以綠色框標示
        \item load label2 : 設定 fracture label,只有在有設定舟骨 label 資料夾時才會生效,因為該 label 依賴舟骨 label 來將相對座標轉為絕對座標
        \item Detection : 將目前影像輸入模型運算。並計算 IoU, Accuracy, Precision, Recall 等指標 (因為我骨折檢測沒做出來,因此顯示的是舟骨區域的指標)

    \end{itemize}

    \paragraph{train.py}
    \begin{itemize}
        \item 訓練偵測舟骨的模型
    \end{itemize}

    \paragraph{tools}
    存放模型的定義、以及一些會用到的函式
    \begin{itemize}
        \item \verb"basic_load.py" : 提供 ui 進行基本的讀取操作
        \item \verb"dataloader.py" : 提供 train 讀取資料集的服務,class 繼承自 torch.dataset
        \item \verb"fasterRCNN.py" : 實現 fasterRCNN,依賴另外定義的部件 ROI 及 RPN 
        \item \verb"tool.py" : 各種供給模型運算的工具函式,例如計算 iou、計算loss、旋轉框轉換...等等
    \end{itemize}
