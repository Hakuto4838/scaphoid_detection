\subsection{class FasterRCNN}
    FasterRCNN 網路
    \paragraph{架構}
    \begin{itemize}
        \item backbone : vgg16
        \begin{itemize}
            \item 載入預訓練模型
            \item 去除最後的分類層
            \item 前10層為基本特徵,不進行梯度計算
        \end{itemize}
        \item RPN 層,參考 class RPN (\ref{clsRPN})
        \item ROIHead 層,參考 class ROIHead (\ref{clsROI})
        \item \verb"num_classes"=2,舟骨及背景
        \item \verb"img_mean" = 0.5,灰階影像進行單通道正規化
        \item \verb"img_std " = 0.3,同上
        \item \verb"min_size" = 1200,對輸入影像大小限制
        \item \verb"max_size" = 1400
       
    \end{itemize}

    \paragraph{forward}
    \begin{enumerate}
        \item 增加靈活性,檢測輸入影像若不為 Tensor 先行轉換
        \item 將影像大小縮放至符合標準。 (訓練時) label bbox 一起轉換
        \item 為了符合 vgg16 的要求,複製單通道成 3 通道
        \item 影像經過 backbone 網路(vgg16) ,產生的特徵圖
        \item 特徵圖經過 RPN ,產生 proposals
        \item 特徵圖與proposals 經過 ROIHead,其包含了最後的分類及回歸,得到最終輸出的框及 scores
        \item (非訓練時) 將提議框 resize ,以對應到未進行縮放的影像
        \item return \verb"rpn_output", \verb"frcnn_output" 以供訓練
    \end{enumerate}

    \paragraph{其他工具}
    \begin{itemize}
        \item \verb"norm_resize_image" 輸入前會進行 resize,因此 frcnn 產出的框不符合原影像座標,此函數以線性插值將框轉換成原圖尺寸
    \end{itemize}