% !TeX root = ./report/report.tex
% 黃柏融 整理

\documentclass[a4paper,12pt]{article}
\usepackage{xeCJK}
\usepackage{graphicx}
\usepackage{float}
\usepackage{subfigure}

\usepackage{pdfpages} %插入PDF

\usepackage{geometry}
\usepackage{titlesec} % 自定义标题样
\usepackage{listings} %取代 verb

\geometry{left=2.5cm, right=2.5cm, top=2.5cm, bottom=2.5cm}

\newfontfamily\frog{NotoSansEgyptianHieroglyphs-Regular.ttf}

\setCJKmainfont[AutoFakeBold=2,AutoFakeSlant=.4]{NotoSansTC-Regular.ttf}

\defaultCJKfontfeatures{AutoFakeBold=2,AutoFakeSlant=.4} %以後不用再設定粗斜
\newCJKfontfamily\Kai{標楷體}       %定義指令\Kai則切換成標楷體
\newCJKfontfamily\Hei{微軟正黑體}   %定義指令\Hei則切換成正黑體
\newCJKfontfamily\NewMing{新細明體} %定義指令\NewMing則切換成新細明體

\newcommand\fish{\frog\symbol{"1319F}}

\titleformat{\section}
  {\Hei \Large} % 使用常規字體,標題不再粗體
  {\fish \,\thesection}{1em}{}

\setlength{\parskip }{0.5em}
%---------------------------------------------------------
\begin{document}
%---------------------------------------------------------
\begin{flushleft} %基本資訊
    \Large \textbf{IMAGE PROCESSING HW2 Report} \\
    \NewMing \normalsize {P76134838 黃柏融} \\
    Submission Date: \today
\end{flushleft}
%---------------------------------------------------------
\tableofcontents

%---------------------------------------------------------
\graphicspath{{./fig/}}
\section{Outline}
以下說明提交的程式碼檔案概要

    \paragraph{ui.py}
    \begin{itemize}
        \item for Demo
        \begin{figure}[h]
            \centering
            \includegraphics[width=0.9\textwidth]{ui.png}
            \label{figui}
            \caption{圖形介面展示}
        \end{figure}
        \item Load Folder : 選擇 image dir,讀取後 Image Display 區域將顯示第一張圖片,並以 [index/Totol image number] 顯示目前的資料夾資料總數、目前影像名稱等等
        \item Previous, Next : 切換上/下一張影像,切換後會自動執行 Detection,在區域中以紅框標示
        \item Load label : 設定舟骨 label 資料夾,若有設定,且路徑中存在與當前影像相同名稱的json,以綠色框標示
        \item load label2 : 設定 fracture label,只有在有設定舟骨 label 資料夾時才會生效,因為該 label 依賴舟骨 label 來將相對座標轉為絕對座標
        \item Detection : 將目前影像輸入模型運算。並計算 IoU, Accuracy, Precision, Recall 等指標 (因為我骨折檢測沒做出來,因此顯示的是舟骨區域的指標)

    \end{itemize}

    \paragraph{train.py}
    \begin{itemize}
        \item 訓練偵測舟骨的模型
    \end{itemize}

    \paragraph{tools}
    存放模型的定義、以及一些會用到的函式
    \begin{itemize}
        \item \verb"basic_load.py" : 提供 ui 進行基本的讀取操作
        \item \verb"dataloader.py" : 提供 train 讀取資料集的服務,class 繼承自 torch.dataset
        \item \verb"fasterRCNN.py" : 實現 fasterRCNN,依賴另外定義的部件 ROI 及 RPN 
        \item \verb"tool.py" : 各種供給模型運算的工具函式,例如計算 iou、計算loss、旋轉框轉換...等等
    \end{itemize}


\section{環境說明}

\paragraph{執行環境}
\begin{itemize}
    \item Python        3.11.11
    \item Pytorch       2.5.1
    \item numpy         2.2.1
    \item opencv-python 4.10.0.84
    \item pillow        11.1.0
    \item matplotlib    3.10.0
\end{itemize}


\paragraph{訓練參數}

\begin{itemize}
    \item lr: 0.001
    \item \verb"num_epochs": 60
    \item 在第 36, 48 epochs 時會下降學習率
\end{itemize}


\begin{figure}[H]
    \centering
    \includegraphics[width=0.9\textwidth]{trainlog.png}
    \label{figTring}
    \caption{訓練過程 loss 趨勢變化}
\end{figure}

\section{模型架構}
\subsection{class FasterRCNN}
    FasterRCNN 網路
    \paragraph{架構}
    \begin{itemize}
        \item backbone : vgg16
        \begin{itemize}
            \item 載入預訓練模型
            \item 去除最後的分類層
            \item 前10層為基本特徵,不進行梯度計算
        \end{itemize}
        \item RPN 層,參考 class RPN (\ref{clsRPN})
        \item ROIHead 層,參考 class ROIHead (\ref{clsROI})
        \item \verb"num_classes"=2,舟骨及背景
        \item \verb"img_mean" = 0.5,灰階影像進行單通道正規化
        \item \verb"img_std " = 0.3,同上
        \item \verb"min_size" = 1200,對輸入影像大小限制
        \item \verb"max_size" = 1400
       
    \end{itemize}

    \paragraph{forward}
    \begin{enumerate}
        \item 增加靈活性,檢測輸入影像若不為 Tensor 先行轉換
        \item 將影像大小縮放至符合標準。 (訓練時) label bbox 一起轉換
        \item 為了符合 vgg16 的要求,複製單通道成 3 通道
        \item 影像經過 backbone 網路(vgg16) ,產生的特徵圖
        \item 特徵圖經過 RPN ,產生 proposals
        \item 特徵圖與proposals 經過 ROIHead,其包含了最後的分類及回歸,得到最終輸出的框及 scores
        \item (非訓練時) 將提議框 resize ,以對應到未進行縮放的影像
        \item return \verb"rpn_output", \verb"frcnn_output" 以供訓練
    \end{enumerate}

    \paragraph{其他工具}
    \begin{itemize}
        \item \verb"norm_resize_image" 輸入前會進行 resize,因此 frcnn 產出的框不符合原影像座標,此函數以線性插值將框轉換成原圖尺寸
    \end{itemize}
\subsection{class RPN}
    \label{clsRPN}
    包含 RPN 網路
    \paragraph{架構}
    \begin{itemize}
        \item anchors size : 64, 128, 256 三種
        \item anchors aspect ratios : 0.5, 1.0, 1.5 
        \item 每個點上共 9 個 anchors
        \item 輸入 vgg16 提取的特徵圖,dim=512
        \item \verb"rpn_conv"一層 $3\times 3$ conv, dim = 512
        \item 輸出結果分別輸入分類層及回歸層,dim 分別為 9、$9\times 4$
       
    \end{itemize}

    \paragraph{forward}
    \begin{enumerate}
        \item 輸入vgg16產生的特徵圖
        \item 經過 \verb"rpn_conv" 並 ReLU 處理
        \item 產生的結果分別以 \verb"cls_layer", \verb"bbox_reg_layer" 產生提議框
        \item 產生的資料轉成提議框:
        \begin{enumerate}
            \item 根據影像大小以及設定的 anchor size 產出所有anchors
            \item 展開輸出,得到一個提議框列表
            \item 將回歸層的輸出套用在原始 anchor,得到真正提議
        \end{enumerate}
        \item 提議框初步篩選
        \begin{itemize}
            \item 分數太低的先濾掉
            \item nms 過濾掉重疊框
        \end{itemize}
        \item (訓練時) 處理 target
        \begin{enumerate}
            \item 根據與 label 的 IOU 判斷提議框為 positive or negative
            \item 計算 label 標出的框與提議框的轉換距離,中心點標記是線性距離,寬高差距取 log
            \item 避免訓練資料中 positive negative 比例懸殊造成失衡,依固定比例進行採樣
        \end{enumerate}
        \item (訓練時) 計算 loss
        \begin{enumerate}
            \item 回歸層使用 \verb"smooth_l1_loss" , $beta= 1/9$ (只計算 positive)
            \item 分類層使用 cross entropy
        \end{enumerate}
        \item return proposals, scores, (訓練時) loss
       
    \end{enumerate}

    \paragraph{其他工具}
    \begin{itemize}
        \item \verb"gen_anchors" : 在特徵圖的每個座標上產生9個 anchor,對應到原始 image 的位置上
        \item \verb"assign_targets_to_anchors" : 判定anchors 是不是背景
        \item \verb"filter_proposals" : 輸入 proposals 與 scores,篩選掉不良的提議 (依大小、分數、重疊)
    \end{itemize}
\subsection{class ROIHead}
    \label{clsROI}
    包含 ROIpooling 以及之後的分類回歸層
    \paragraph{架構}
    \begin{itemize}
        \item \verb"num_classes=2", 分類背景或是舟骨
        \item \verb"in_channels=512",來自vgg16 產生的特徵圖
        \item \verb"fc_inner_dim=1024",兩層 fc
        \item 一個 \verb"cls_layer",一個 \verb"bbox_reg_layer"
        
    \end{itemize}

    \paragraph{forward}
    \begin{enumerate}
        \item 輸入特徵圖(dim = 512)、proposals
        \item (訓練時) 處理 target,控制 positive 提議框佔總target 比例,以免資料失衡
        \item \verb"roi_pool", 將 proposals 框出的區域剪切並 pool 成同樣大小
        \item 平坦層 $\rightarrow fc \times 2$
        \item 分別用分類層與回歸層取得最後輸出
        \item (訓練時) 計算 loss
        \begin{itemize}
            \item 分類 : \verb"cross_entropy"
            \item 回歸(限非背景框) : \verb"smooth_l1_loss"
            \item \verb"return cls_loss, loc_loss"
        \end{itemize}
        \item (非訓練時) : 
        \begin{enumerate}
            \item 將回歸結果套用在 proposals 上
            \item 分類結果使用 softmax 歸一
            \item 刪除分類為背景的框
            \item 過濾過小、分數過低、NMS 篩選
            \item 依據分數高低(信心程度)排序框,return boxes, scores, labels
        \end{enumerate}
    \end{enumerate}

\section{骨折檢測紀錄(我最後沒做出來,紀錄一下)}

\subsection{加入旋轉角}

原本我打算修改 Frcnn 的架構,加入旋轉角作為提議框維度,也在 tool 寫出將座標轉換成五個維度 (中心x,y、寬高、旋轉)。

但是修改到計算 roi 那關時遇到困難,無法有效計算兩個旋轉提議框的重疊面積。之後只好放棄

\subsection{使用 mmrotate 檢測}

之後嘗試使用開源的 mmrotate 模型建構檢測骨折的模型。但是搞了很久連環境都安裝不起來。因為截止時間快到我還沒製作 ui ,因此決定放棄後半部分的骨折檢測,至少將舟骨檢測的功能做出 ui 來 demo

%---------------------------------------------------------
\end{document}